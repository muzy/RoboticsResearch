\documentclass[twocolumn]{article}

\usepackage[english]{babel} 

\usepackage[T1]{fontenc}
\usepackage[utf8]{inputenc}
\usepackage{graphicx}

% Math

\usepackage{amsmath}
\usepackage{amssymb}
\usepackage{amsthm} 

\newtheoremstyle{break}
   {\topsep}{\topsep}%
   {\itshape}{}%
   {\bfseries}{}%
   {\newline}
   {\thmname{#1}\thmnumber{\@ifnotempty{#1}{ }\@upn{#2}}%
    \thmnote{ {\bfseries(#3)}}}% 
    
\newtheorem{definition}{Definition}

\usepackage[parfill]{parskip}
%\usepackage[onehalfspacing]{setspace}
\usepackage{newclude}

%\usepackage[sortcites=true, style=authoryear,natbib=true]{biblatex}
\usepackage{biblatex}
\bibliography{bibliography}

%Autornamen in Bibliography fett
%\AtBeginBibliography{\renewcommand*{\mkbibnamelast }[1]{\textbf{#1}}}
%\AtBeginBibliography{\renewcommand*{\mkbibnamefirst }[1]{\textbf{#1}}}

%\makeindex

\title{Development of a educational and sustainable robotic platform for children and adults}
\date{\today}
\author{Sebastian Muszytowski \\Baden-Wuerttemberg Cooperative State University Mannheim }

\begin{document}

\twocolumn[
        \maketitle
        \begin{@twocolumnfalse}
        \maketitle
        \end{@twocolumnfalse}
]


\section{Introduction}
Getting children in touch with technology is important in todays technologically advanced society. In schools technology (specifically electrical engineering and programming) is often taught using robots. Robots are well suited for children since they allow interaction with the reality in contrast to arbitrary technology related exercises such as programming execises which doesn't affect the real world.\newline 
Since schools are required to save money, robots used for teaching stay property of the schools. Therefore children usually cannot take the robot home to work with it. In addition modification of robot is usually not permitted which reduces potential learning experiences. Giving students the full control over all possibilities of the robot requires them to own the robot. This requires to robot to be cheap, extendable and suitable for educational purposes.\newline
The development of a robot plattform for educational purpose requires requirement engineering, market research and evaluation of robots which are already used in education. Once the research is completed a new robot plattform can be constructed taking the research results into account.
\section{Requirement Analysis / Project Goals}
The target user group of the robot are mainly children and adults with very little or no experience in electrical engineering and programming. Teaching the knowledge requires a platform which is designed to be educational. Therefore it has to comply with several requirements which affect technical aspects as well as social aspects.
The following lists describes the basic requirements on which existing platforms can be assessed. The assessent critera can also be used to make decisions when developing a new platform.
\begin{description}
\item[Affordable] \hfill \\ The robot platform must be affordable for everyone. Especially when the robot is used in schools every child should own the robot to increase the possibilities to modify and extend it. Robots which stay property of a school may result in social problems since socially disadvantaged children cannot afford their own robot whereas socially advantaged children can. A considerable price tag for a educational robot is the educational budget for a wellfare recipient which is 100 Euro per year in Germany as defined defined in SGB II §28 (as of the 07.05.2013).
\item[Educational] \hfill \\ Documentation, learning materials and suggestions for lessons are needed to teach the usage of the robot. The robot should be designed to teach both electrical engineering as well as programming. This can be realized by having exercises specific to the topic e.g. by soldering the robot first with programming afterwards. By using strategies like Poka-yoke (mistake proofing) the platform becomes more robust and therefore can prevent frustration.
\item[Sustainable] \hfill \\ Sustainability is an environmental issue which should be taken into account. Sustainability includes serveral aspects such as the choice of materials, their durability and the repairability. It is a huge bonus if materials are environment-friendly and produced in a socially acceptable manner. If the robot can be repaired or interchanged using household or easy to get items the robot can be considered sustainable. 
\item[Extendable] \hfill \\ Extendabilitiy enables the robot to adapt to new tasks. Interchangeable sensors are the key to a flexible plattform. Unused microcontroller pins should be mapped onto easy accessible pins which require no soldering.
\item[Open-Source] \hfill \\ Open-Source products can be easily modified, extended and remixed into new, continously improved, products.  
\end{description}


\section{State of the Art}
The development of a new robot platform requires the assessment of current robotic platforms using the requirements developed. In focus of the assessment are popular robot platforms which are either actively used in schools or developed with education in mind. All of the platforms are independently developed and focus on different ideas. 

\subsection{Asuro}
\begin{figure}[h!]
  \centering
  \includegraphics[width=0.5\textwidth]{images/asuro.jpg}
  \caption{Picture of the Asuro (CC-BY-SA Robin Gruber)}
\end{figure}

The Asuro robot is developed by the German Aerospace Center (Deutsches Zentrum für Luft- und Raumfahrt e.V.) with focus on edcuational use. Asuros development started in 2004 and is now discontinued. The robot is shipped as a do-it-yourself kit and requires about eight hours build time for a novice. 

\subsection{Arduino Robot}
\begin{figure}[h!]
  \centering
  \includegraphics[width=0.5\textwidth]{images/arduinorobot.jpg}
  \caption{Picture of the Arduino Robot}
\end{figure}

The Arduino Robot is the first robot created by the arduino foundation. It comes preassembled and contains two microcontrollers 
\subsection{Lego Mindstorms}
\begin{figure}[h!]
  \centering
  \includegraphics[width=0.5\textwidth]{images/mindstorms.jpg}
  \caption{Picture of the Lego Mindstorms NXT (from http://www.devoxx.com/)}
\end{figure}
\subsection{Thymio II}
\begin{figure}[h!]
  \centering
  \includegraphics[width=0.5\textwidth]{images/thymioii.jpg}
  \caption{Picture of the Thymio II (from https://aseba.wikidot.com/en:thymio)}
\end{figure}


\section{Case studies}
A robot is built using parts and modules which can be grouped into functional groups like movement, sensors or communication. The goal of each functional group can be achieved in various ways which have to be evaluated using the requirements used in the evaluation of other robot platforms. The difficulty in the evaluation process are dependencies between different modules, e.g. when a sensor has special power requirements. 
\subsection{Movement}
Movement is the most important interaction with the robots environment. In contrast to a stationary robot a moving robot can interact with a wider area but requires battery power. Different approaches for robot movement exist which will be part of critical evaluation.
\subsubsection{Wheels}
Wheeled robots are common since their mechanic and control is rather simple. Typically robots either have two 
\paragraph{Wheeled with two DC motors}

Wheeled with two Stepper motors
Omniwheels

\subsubsection{Legs}
Bipod
Qadropod
Hexapod

\subsection{Sensors}
\subsection{Communication}
\subsection{Microcontroller}
\subsection{Power Supply}

\section{Robot Construction}



\nocite{*}

\printbibliography[maxnames=25]
\addcontentsline{toc}{part}{References}
\end{document}
