\chapter*{Zusammenfassung}
Kinder, Jugendliche und Erwachsene an Technik heranzuführen ist eine wichtige Aufgabe in einer zunehmend technisierten Gesellschaft. Ein verbreiteter Ansatz ist die Heranführung mit Hilfe von Robotern da diese ihre Umgebung wahrnehmen und mit dieser interagieren können. 

Eine Evaluation zeigt, dass viele Robotikplattformen für den Bildungssektor Kriterien an eine nachhaltige und pädagogisch wertvolle Lösung zu Teilen bereits erfüllen. Kriterien wie günstige Anschaffungskosten, Erweiterbarkeit und geringe Komplexität sind nur zu Teilen erfüllt und zeigen die Notwendigkeit einer neuen Plattform.

Nach einer detalierten Analyse zu Bewegungskonzepten, Sensoren, Kommunikation, Mikrocontroller-Plattformen und Batterie-Systemen wird anhand einer Auswahl von Komponenten eine kostengünstige, pädagogisch wertvolle, quelloffene und erweiterbare Robotikplattform konstruiert. Eine detalierte Darstellung der Entwicklungsschritte zeigt auf, welche Designentscheidung die Bewertungskriterien positiv beeinflussen.

Die Abschließende Evaluation von Chancen und Risiken zeigt, dass eine pädagogisch sinnvolle und wirtschaftliche Verwendung der neu entwickelten Robotikplattform möglich ist. Die zuvor auf andere Plattformen angewendeten Evaluationskriterien zeigen eine deutliche Verbesserung in Punkten Erweiterbarkeit, Anschaffungskosten und Nachhaltigkeit.

\chapter*{Abstract}
Getting children, young adults and adults in touch with technology is important with today’s technologically advancing society. A common approach is to use Robots since they can sense their environment and modify it. 

The evaluation of different robotic platforms designed for educational use shows that criteria such as sustainability and educational value are met to a certain extend. Other criteria such as low cost of purchase, extendibility and low complexity are partially met and therefore show the need for a new robotic platform.

After a detailed analysis of movement concepts, sensors, communication techniques, microcontroller platforms and power supply systems the optimal components for the new  platforms were chosen. The choice is based on cost, educational, extension and open source criteria. A detailed description of the construction steps allows to comprehend design decisions which positively influence the evaluation criteria.

A evaluation of chances and challenges on the constructed robot shows that a educational use of the platform is possible whilst operating economically. The previously used evaluation criteria for evaluating other platforms applied to the new platform shows enhancements in extendibility, purchase costs and sustainability.

